\documentclass[12pt]{article}
\usepackage[margin=3cm]{geometry}
\usepackage[utf8]{inputenc}
\usepackage{indentfirst}
\usepackage{amsmath}
\usepackage{fancyhdr}

\fancyhf{}
\pagestyle{fancy}


\renewcommand{\headrulewidth}{0pt}
\renewcommand{\footrulewidth}{1pt}
\renewcommand\footnoterule{}

\rfoot{\thepage}
\lfoot{Afonso Gonçalves, Daniel Seara}


\begin{document}

\title{Análise e Síntese de Algoritmos - 1º Projeto}
\author{Grupo 16 \\ Afonso Gonçalves - 89399 \\ Daniel Seara - 89427}
\date{}
\maketitle


\thispagestyle{fancy}


\section{Introdução}

O primeiro projeto da cadeira de ASA teve como objetivo executar uma auditoria a uma rede. Esta rede poderá estar dividida em sub-redes, 
sendo que temos de realizar a auditoria a todas elas. Para tal é necessário calcular que routers da rede, ao serem desligados, aumentariam o número de
sub-redes.\par

O Input é dado com M+2 linhas. A primeira com o número de routers da rede, a segunda com o número (M) de ligações entre routers e as restantes M linhas que 
representam uma ligação entre 2 routers. Esta ligação é bidireccional, ou seja, uma ligação do router $x$ para o router $y$ implica também de 
$y$ para $x$.\par

O Output é dado por 4 linhas. Primeiramente, é apresentado o número de sub-redes da rede fornecida. Seguidamente, são apresentados os identificadores de cada 
uma das sub-redes. Depois, é apresentado o número de routers cuja desconecção levariam a um número maior de sub-redes. Por último, é apresentado o número de routers 
da maior sub-rede resultante da remoção dos routers que quebram uma sub-rede.\par

\section{Descrição da Solução}


    % Linguagem utilizada
    Decidimos usar a linguagem C++ para a resolução deste problema,
    por ser uma linguagem eficiente, por ter uma boa biblioteca de estruturas de dados 
    que consideramos necessário e ainda por constituir um desafio começar a trabalhar 
    com uma nova linguagem, tão usada e potente nos dias de hoje.

    % Contrução da Solução
    \bigskip
    Identificámos que este problema pode ser resolvido como um problema de grafos:
    Cada router é representado por um vértice e as ligações entre estes pelas respetivas 
    arestas. Os routers que podem quebrar a rede são os vértices de corte ($AP$)
    do grafo e as sub-redes os seus subgrafos.
    \par
    Começámos por desenhar um algoritmo que identificava as pontes do grafo, uma vez
    que, removendo pelo menos um dos vértices onde esse arco incide, criaríamos novos 
    subgrafos. Teríamos especial atenção aos vértices que apenas tivessem uma aresta,
    pois a sua remoção não aumentaria o número de subgrafos. Esta abordagem não contempla
    os vértices de articulação que não formam pontes (Figura 1), por isso mudámos de estratégia.
    \par
    Em aula teórica e com alguma pesquisa na internet
    %TODO: fazer referência bibliográfica
    \footnote{https://www.geeksforgeeks.org/articulation-points-or-cut-vertices-in-a-graph/}
    \footnote{https://www.hackerearth.com/practice/algorithms/graphs/articulation-points-and-bridges/tutorial/}
    , aprendemos que o algoritmo de 
    Tarjan permite encontrar os $AP$'s dos grafos. Como esse algoritmo executa uma DFS, 
    seria possível calcular simultaneamente o número de sub-redes do grafo. Para além disso,
    ao visitar cada subgrafo de uma vez, conseguimos também saber o maior ID que lhe pertence.
    \par
    Numa nova iteração do algoritmo, passámos a percorrer os vértices por ordem decrescente de 
    ID, uma vez que estes serão os maiores do respetivo subgrafo. Deste modo, obtemos ainda os 
    ID's máximos de cada subgrafo por ordem decrescente, não sendo necessário ordená-los.
    \par
    Para calcular o número de vértices do maior subgrafo resultante da 
    remoção de todos os vértices críticos encontrados, começámos por  
    remover todos os $AP$'s e as suas arestas e executar uma DFS.
    Rapidamente nos apercebemos que seria uma solução demasiado cara para o efeito: O uso de uma
    flag que indicasse em tempo constante se um vértice é um $AP$ permite ignorá-lo durante a DFS,
    como se tivesse sido removido.
    \par
    % TODO: FALAR DO GRAFO 100% LIGADO PARA O CASO OMEGA(1)





\end{document}
