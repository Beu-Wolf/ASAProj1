\documentclass[12pt]{article}
\usepackage[margin=3cm]{geometry}
\usepackage[utf8]{inputenc}
\usepackage{indentfirst}
\usepackage{amsmath}
\usepackage{fancyhdr}

\fancyhf{}
\pagestyle{fancy}

\renewcommand{\headrulewidth}{0pt}
\renewcommand{\footrulewidth}{1pt}

\rfoot{\thepage}
\lfoot{Afonso Gonçalves, Daniel Seara}


\begin{document}

\title{Análise e Síntese de Algoritmos - 1º Projeto}
\author{Grupo 16 \\ Afonso Gonçalves - 89399 \\ Daniel Seara - 89427}
\date{}
\maketitle


\thispagestyle{fancy}


\section{Introdução}

O primeiro projeto da cadeira de ASA teve como objetivo executar uma auditoria a uma rede. Esta rede poderá estar dividida em sub-redes, 
sendo que temos de realizar a auditoria a todas elas. Para tal é necessário calcular que routers da rede, ao serem desligados, aumentariam o número de
sub-redes.\par

O Input é dado com M+2 linhas. A primeira com o número de routers da rede, a segunda com o número (M) de ligações entre routers e as restantes M linhas que 
representam uma ligação entre 2 routers. Esta ligação é bidireccional, ou seja, uma ligação do router $x$ para o router $y$ implica também de 
$y$ para $x$.\par

O Output é dado por 4 linhas. Primeiramente, é apresentado o número de sub-redes da rede fornecida. Seguidamente, são apresentados os identificadores de cada 
uma das sub-redes. Depois, é apresentado o número de routers cuja desconecção levariam a um número maior de sub-redes. Por último, é apresentado o número de routers 
da maior sub-rede resultante da remoção dos routers que quebram uma sub-rede.\par



\end{document}